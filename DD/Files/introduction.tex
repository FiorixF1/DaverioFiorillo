\subsection{Purpose}

The purpose of this derivable is to provide an architectural design solution for Travlendar+ that fulfils all functional and non-functional requirements expressed in the linked Requirement Analysis and Specification Document.

The following document will also provide an overview of an implementation, integration and test plan solution, bounded with this derivable.

The target audiences of this paper are developers, testers teams and analysts involved in the project.

\subsection{Scope}

Travlendar+ is a calendar-based application thought to support user thorough his appointment scheduling and transport management.

The system will provide to user:

\begin{itemize}
	\item the possibility to insert scheduled appointments
	\item the insertions, modification and organization of different types of preferences
	\item registration and login in order to access backup/sync options, beside a more advance preferences management
	\item different route choices to reach the location of appointments
\end{itemize}

(for further details look at RASD document).

The system will offer his services to user through a Web GUI and a mobile application supporting different operating systems and device format.



\subsection{Definitions, Acronyms, Abbreviations}


\begin{itemize}
	
	
	
	\item Appointment: location the user has to reach at a fixed date and time
	\item Route: journey between two appointments, it is composed by a series of paths
	\item Path: part of a route traversed with a specific mean of transport
\end{itemize}

\subsubsection{Acronyms}
\begin{itemize}
	\item \textsl{RASD}: \textbf{R}equirement \textbf{A}nalysis and \textbf{S}oftware \textbf{S}pecification
	\item \textsl{DD} \textbf{D}esign \textbf{D}ocument
	\item \textsl{GUI}: \textbf{G}raphic \textbf{U}ser \textbf{I}nterface
	\item \textsl{WebUI}: \textbf{W}ew \textbf{U}ser \textbf{I}nterface
	\item \textsl{API(s)}: Application Program Interface(s)
	\item \textsl{DBMS}: \textbf{D}ata \textbf{B}ase \textbf{M}anagement \textbf{S}ystem
\end{itemize}
\subsubsection{Abbreviations}
\begin{itemize}
	\item \textsl{MA}: \textbf{M}obile \textbf{A}pplication
	\item \textsl{WebApp}: Web Application
	
\end{itemize}

\subsection{Revision History}

%Revision History

\subsection{Reference Document}

\begin{itemize}
	\item \textsl{Travlendar+ RASD document}
	\item \textsl{"Mandatory\_Project\_Assignment.pdf" }: assignment given of the project
	
\end{itemize}

\subsection{Document Structure}



	\paragraph*{1 - Introduction} 
	The first section offers an overview on the content of the following document, highlighting the purpose of this derivable and recalling a brief description of the problem itself. It also contains references to other documentation linked to this project.

	\paragraph*{2 - Architectural Design} 
	Firstly, he architectural design section presents an overview of a proposed system architecture to accomplish RASD specification. 
	\newline
	Secondly, it steps into the system identifying behaviour of components, their interfaces and interoperability with other components, inside or outside the system. 
	\newline
	Finally, it explains the thought behind design choices and it gives a list of patterns employed.

	\paragraph*{3 - Algorithm Design}
	This section provides a list of the most significant algorithms, used by system, expressed in object oriented pseudo-code. They are given in order to specify some critical operation steps.

	\paragraph*{4 - User Interface Design}
	This section offers a look upon the user interfaces in order to give a good representation of how the UI will look like to users by mean of interfaces mock-up. Furthermore, it' offered a deeper inquiry on the user interaction with the system through UX and BCE diagrams.

	\paragraph*{5 - Requirement Traceability}
	Section implied in the traceability purpose of requirements, defined in RASD document, with components identified by current derivable in order to increase the observability of requirements fulfilment in following parts of system development and testing.

	\paragraph*{6 - Implementation, Integration and Test Plan} 
	It defines the strategy and provides a sequential plan for the implementation, integration and test processes, describing the sequence in which components are developed, integrated and tested together.

	\paragraph*{7 - Effort Spent}
	Appendix showing the commitment required by the project to the team.

