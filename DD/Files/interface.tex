\subsection{Mock ups}


The following mock-ups give an idea of how the mobile user interface will be structured. The software used for realizing these mock-ups is Pencil, which doesn't produce very nice looking interfaces, but it is helpful for creating a structure that will be followed when realizing the real user interface.

\begin{figure}
\centering
\includegraphics[scale=0.5]{{"Images/UI - Login"}.png}
\caption{\label{fig:metamodel2}Login screen}
\end{figure}

\begin{figure}
\centering
\includegraphics[scale=0.5]{{"Images/UI - Signup"}.png}
\caption{\label{fig:metamodel2}Sigup screen}
\end{figure}

\begin{figure}
\centering
\includegraphics[scale=0.5]{{"Images/UI - Home"}.png}
\caption{\label{fig:metamodel2}Home screen with calendar}
\end{figure}

\begin{figure}
\centering
\includegraphics[scale=0.5]{{"Images/UI - Add appointment"}.png}
\caption{\label{fig:metamodel2}Add an appointment}
\end{figure}

\begin{figure}
\centering
\includegraphics[scale=0.5]{{"Images/UI - Add route"}.png}
\caption{\label{fig:metamodel2}Add a route}
\end{figure}

\begin{figure}
\centering
\includegraphics[scale=0.5]{{"Images/UI - Preferences"}.png}
\caption{\label{fig:metamodel2}Set preferences}
\end{figure}

\begin{figure}
\centering
\includegraphics[scale=0.5]{{"Images/UI - Select route"}.png}
\caption{\label{fig:metamodel2}Select a route}
\end{figure}

\clearpage

\newpage

\subsection{UX diagram}

The following UX diagram shows how a user of the application would navigate through the different screens while using the available features. Each screen is represented as a class with stereotype «screen» and contains the possible functionalities, while the classes with stereotype «input form» contains attributes that the user must provide inside a specific screen. The diagram has been created following the structure of the mockups in the previous paragraph.

\begin{sidewaysfigure}
\centering
\includegraphics[width=\textwidth]{{"Images/UI - UX Diagram"}.png}
\caption{\label{fig:metamodel2}UX Diagram}
\end{sidewaysfigure}