\subsection{Purpose}
The following document is the Requirement Analysis and Specification Document (RASD) for the formalization and description of all the needed features, constraints and recommendations that will constitute the proposed system.

The paper will focus on the elicitation and analysis of all functional and nonfunctional requirements, also verified by automated logic verification software (Alloy Analyzer [MIT]) and supported by UML schemes and scenarios.
The provided document will also include a concise analysis of the environment and it tries to clarify all the interaction among system parts and external world.

The target audiences of this document are all the stakeholders involved in the development of the system and it can be used as contractual base for the project.


\subsection{Scope}

\subsubsection{System description}

The system that would be developed is a calendar-based application that provides support for planning of appointments, automating the travel arrangement process according to the user preferences.
The system will be able to plan travel routes choosing transport options among public and private means, including trains, trams, taxis, bicycles, bike and car sharing, owned automobile and others. Therefore, the system will provide a set of possible route options, trying to minimize travel times, route length, carbon footprint or number of changes.

Also, it will have to recollect travel informations on public transportations and travelling conditions from different external sources.
Beside, the application will give the possibility to users to register themselves, allowing them to access a series of extra features, including the possibility of setting personal preferences and backup personal agenda and routes.

User preferences include the possibility to set travel pauses, flexible launch (30 minute of travel pause between 11.30-14.30), enabling and disabling certains types of transportation means in a specific period of the day or anytime and in addition it will give the possibility to user to create profiles of preferences and link them to single appointments)
Finally, system should be capable to interacting with any type of external public transportation service, although it will focus on make available them in the city of Milan.

\subsubsection{World Phenomena}

Many public transport service companies offers the possibility to obtains informations on timetables, routes, stations and status of services in aggregate way providing public Application Public Interfaces (APIs). Thus it is guaranteed the theoretical possibility to make queries in any moment and get the latest valid informations on transports.

Others doesn’t offer the same possibilities; for instance Trenitalia hasn’t yet a public API, but there are different opensource projects that allow to fill the gap.

Moreover, a lot of web mapping services makes available public APIs too. In this case it is possible to get GPS location from an address or vice-versa, or update a portion of map and even process a route given a starting point and a destination.
All the required supported companies provide aforementioned services.

\subsubsection{Goals}
%what you write here is a comment that is not shown in the final text
\begin{itemize}
	\item[G1] Allow the user to add an appointment
	\subitem[G1.1] The user can add the date of the appointment through a calendar
	\subitem[G1.2] The user can select the location through a map
	\subitem[G1.3] The appointment must be processed by the system
	\item[G2] - Provide a route to the user for reaching the appointments
	\subitem[G2.1] The user must reach on time his/her appointments
	\subitem[G2.2] The user can choose the starting location and time of the route
	\subitem[G2.3] Generate routes according to the preferences of the user
	\subitem[G2.4] The application provides a route for each objective, minimizing each of these attributes: length, duration, number of changes, carbon footprint
	\subitem[G2.5] Always provide a route
	\subitem[G2.6] During strike days, public transportation must not be available
	\subitem[G2.7] Report unfavorable weather conditions
	\subitem[G2.8] Update unfavorable weather conditions
	\subitem[G2.9] The user can save one route among the shown ones
	\item[G3] Allow the user to sign up into the application
	\subitem[G3.1] The registration must allow the univocal recognition of the user
	\item[G4] Allow the user to log in
	\subitem[G4.1] The system allow the login through e-mail and password
	\subitem[G4.2] The application allow the login through telephone number
	\item[G5] Allow the user to add his own preferences
	\subitem[G5.1] The user must be logged
	\subitem[G5.2] The preferences are synchronized between the application and the database
	\subitem[G5.3] The user can chose the available transport means
	\subitem[G5.4] The user can add a priority to the means of transport
	\subitem[G5.5] The user can add the maximum length of routes on foot or by bicycle
	\subitem[G5.6] For each vehicle the user can choose a time slot of validity
	\subitem[G5.7] The user can set Flexible Lunch
	\subitem[G5.8] The user can add breaks for fixed moments of the day
	\subitem[G5.9] The user can add a private car or bicycle
	\subitem[G5.10] The user can organize his preferences in “Preferences Profiles"
	\item[G6] Allow the user to manage his account
	\subitem[G6.1] The user must be able to remove appointments and routes
	\subitem[G6.2] The user must be able to modify appointments and routes
	\subitem[G6.3] The user must be able to delete his account
	\item[G7] Allow the user to report events and disservices 
	\subitem[G7.1] The user can report strikes using the application
	\subitem[G7.2] The user can report faults, malfunctions and suggestions
\end{itemize}

\subsection{Definitions, Acronyms, Abbreviations}

\subsubsection{Definitions}
\begin{itemize}
	\item API(s) : Application Program Interface(s)
	\item RASD : Requirement Analysis Specification Document
	\item Appointment: location the user has to reach at a fixed date and time
	\item Route: journey between two appointments, it is composed by a series of paths
	\item Path: part of a route traversed with a specific mean of transport
\end{itemize}


\subsubsection{Acronyms}
\begin{itemize}
	\item UI: User Interface
\end{itemize}

\subsubsection{Abbreviations}
\begin{itemize}
	\item MA (Mobile Application): part of the system
	\item PT: Public Transportation
\end{itemize}

\subsection{Revision history}




\subsection{References}


\begin{itemize}
	\item Specification Document assigned: \texttt{“Mandatory\_Project\_Assignments.pdf”}
	\item Software Requirements Specification Guidelines : IEEE 830-1998
	\item Alloy Website: \url{http://alloy.mit.edu/alloy/}
\end{itemize}

\subsection{Document Structure}

The document is composed of six different parts.

\begin{enumerate}
	
	\item The introduction has the objective to support the reader into the reading of the document providing a brief description of the problem and the actual technological landscape description. It also include a list of definition, abbreviations and acronyms commonly employed through the rest of the document.
	\item The second part offers an overall description of the system, then focused on the boundaries that divide the system from world, especially inspecting all interactions on shared phenomena. In addition, it is provided a short description of users. FInally, it is listed and described a set of core functionalities that will be supported and assumptions.
	\item The third part provides a more complex and detailed description of functionalities, also in terms of requirement. This part is addressed specifically to development team
	\item This section provides an Alloy model for verification of goals through a formal analysis of requirements and domain assumption. This is beneficial in assuring the consistency of the model. It also provide a representation of the world and system.
	\item The fifth part provides the effort spent in term of time involved into the project by the authors.
	In the last part, there are references to external sources.
	
\end{enumerate}

